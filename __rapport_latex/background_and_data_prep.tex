% !TEX root = rapport_root.tex
\section{Background}

\subsection{Previous work}

This project is inspired by the ongoing work of Helene Lane, where she employs a
centroids based natural language processing (NLP) method developed by
\textcite{Odden2024Using}. Centroids for text summarization were first
introduced by \autocite{Radev2004Centroid} and consist of averaging the
positions of a set of samples in the embeddings space. These averages -- the
centroids -- should then represent some semantic meaning shared by its
constitutive samples.

Lane has embedded a dataset of 1222 whole articles from PRPER. From these,
centroids were computed based on handpicked sets representing common topics in
PRPER. The topic categories are: ``Mechanics'', ``Electricity and Magnetism'',
``Sound and Waves'', ``Relativity'', ``Thermal Physics'', ``Optics'', ``Fluid
Dynamics'', ``Quantum Physics'', ``Astrophysics'', ``Identity'', ``Lab'' and
``Attitudes''. By calculating the distance from each of the articles to these
centroids, she can visualize the distribution of topics within physics education
in an embedding (or ``meaning'') space. She intends to use the distribution to
investigate the evolution of topics in PRPER over time. In other words, like
\autocite{Odden2024Using}, she employs text embeddings and centroids to conduct
a qualitative analysis.


\subsection{Project Goals and Structure}

A limitation of Lane's work is its reliance on \emph{article-level embeddings},
which reduces each complex document to a single point in the embedding space.
Since scientific articles have a clear sectioned structure (e.g., theory,
methods, results), a more fine-grained, \emph{section-level analysis} could reveal
new features.

To address this, our project comprises four primary components that build upon
one another:
\begin{itemize}
    \item \textbf{Data Preparation:} To process the raw XML data by splitting
    each of the 1222 articles into its constituent sections and converting the
    text of each section into a high-dimensional vector representation
    (embedding).

    \item \textbf{Exploratory Analysis:} To investigate the structure of the
    resulting embedding space using dimensionality reduction and clustering
    techniques, identifying the primary factors that influence a section's
    position (e.g., article theme vs. section type).

    \item \textbf{Section Type Classification:} To develop and evaluate multiple
    models—including heuristics and large language models for classifying each
    section according to its rhetorical function (e.g., `Introduction`,
    `Methods`, `Results`).

    \item \textbf{Fine-Grained Trend Analysis:} To leverage the classified
    sections to identify the specific theoretical frameworks and research
    methods used within the articles and to analyze their prevalence over the
    20-year history of the journal.
\end{itemize}


\section{Article Splitting and Embedding}

The initial splitting and embedding were in many ways straightforward. Despite
being spread over 20 years of publishing, the data had a predictable XML
structure. The extracted data could then be passed to an embedding function to
yield a vector representation. The article splitting algorithm resulted in 7313
sections from the 1222 articles.

We chose to use the closed source ``voyage-large-2" model from Voyage AI for our
initial embedding, choosing a 1024 dimensional output. At the time of writing,
Voyage AI's models are considered to provide the best embeddings for general
purposes. We therefore chose one of their models for our initial embeddings that
were to be used for general data exploration. A viable open-source alternative
would be Jina AI's embedding models.

To see the implementation of the data extraction and embedding, see the
``pre-processing.ipynb'' computational notebook.


\subsection{Transformer Fine-tuning}
In recent years, transformer-based language models have become the industry standard for downstream classification tasks, typically used in combination with a final set of classification layers, fine-tuned on domain-specific datasets. In our analysis we employ BERT \parencite{devlin2019bertpretrainingdeepbidirectional}, a sophisticated LLM utilizing attention mechanisms introduced by \textcite{vaswani2023attentionneed} to capture context-specific representations of textual data. More precisely we use SciBERT, a BERT-based language model trained on a large corpus of scientific publication, offering state-of-art performance on downstream scientific NLP tasks \autocite{Beltagy2019SciBERTAP}. Fine-tuning is done by adding task-specific layers of classification heads, iteratively adjusted via supervised learning. This is implemented using Hugging Face's Transformer library \parencite{wolf-etal-2020-transformers}.

\subsection{Multi-layer Perceptrons}
Multi-Layer Perceptrons (MLPs) are a class of feed-forward neural networks generally consisting of an input and output layer sandwiching one or several hidden layers \autocite{HORNIK1989359}. Layers consist of a finite number of nodes where each node is fully connected to the previous and subsequent layer, with individual neuron outputs dictated by non-linear activation functions. Despite their simple structure, MLPs are capable of modeling complex non-linear relationships \autocite{HORNIK1989359} and are widely used for classification tasks. In this paper, MLPs are used as components in custom hybrid classification models.